\documentclass[a4paper]{article}

%% Language and font encodings
\usepackage[utf8x]{inputenc}
\usepackage[T1]{fontenc}

%% Sets page size and margins
\usepackage[a4paper,top=3cm,bottom=2cm,left=3cm,right=3cm,marginparwidth=1.75cm]{geometry}

%% Useful packages
\usepackage{amsmath}
\usepackage{graphicx}
\usepackage[colorinlistoftodos]{todonotes}
\usepackage[colorlinks=true, allcolors=blue]{hyperref}

\title{PRO : Cahier des charges}
\author{Groupe 1-B}
\date{Mars 2018}

\begin{document}

\maketitle

\paragraph{Environnement de développement envisagé}
\begin{enumerate}
\item Langage de programmation : Java
\item Base de données : MySQL
\item Librairie graphique : JavaFX
\item API tierces : GSON, JUnit
\item Framework library for client server : Netty, Apache MINA, CoralReactor, autre ?
\end{enumerate}

\paragraph{Fonctionnalités de base}
\begin{enumerate}
\item Affichage des salles libres sous forme de plan selon les paramètres suivants, choisis par l'utilisateur :
	\begin{enumerate}
	\item Position de l'utilisateur
	\item Bâtiment : St-Roch ou Cheseaux
	\item Numéro d'étage
	\item Horaire, à choix : maintenant, ou sur une ou plusieurs plages horaires
	\item Présence de prises électriques
	\item Présence de tableaux
	\item Présence d'ordinateurs
    \end{enumerate}
\item Affichage de divers commodités sur le plan d'étage : toilettes, Selectas et machines à café
\end{enumerate} 

\paragraph{Fonctionnalités facultatives}
\begin{enumerate}
\item Mode daltonien
\item Mode nuit
\item Génération de fichiers .csv ou .txt contenant les salles libres
\item Possibilité à l'utilisateur d'indiquer si une salle est utilisée, soit par user input soit par détection de la borne wi-fi utilisée.
\end{enumerate} 

\end{document}

